\documentclass[a4paper,norsk]{article}
\usepackage[utf8]{inputenc}
\usepackage[T1]{fontenc,url}
\usepackage{babel,textcomp}
\usepackage{graphicx, wrapfig}
\usepackage{graphics}
\graphicspath{
	{Code/figs/}
	{Code/scorefigs/}
}
\usepackage{cite}
\usepackage{amsmath}
\usepackage{bm}
\usepackage{stackengine}
\usepackage{listings}
\usepackage{amsfonts}
\urlstyle {sf}
\title {Project 2 FYS-STK4155 Autumn 2018}
\author {Jon Audun Baar}
\begin{document}
\maketitle

\section{Introduction}
The one sentence about what machine learning is for the new guy.

The concept of machine learning have gained hughe popularity over the 
last couple of years. Different machine learning techniques 
have a wide range of applications and can be a major asset if you know when 
to use what. When to use what is exactly what we're going to have a brief
peek into in this project.

The last couple of years different machine learning techniques have
gained enormous popularity due to several factors. 

Two widely used machine learning methods
are logistic regression and neural networks. In this project we aim to
evaluate the performance of these two methods on an often studied problem
in physics.
\par
First we use both methods to predict the energy of the system. Then we 
apply both methods over again to predict the phases of the systems.
\section{Method}

\subsection{Logistic Regression}

\subsection{Neural Networks}

\subsection{Performance measures}

\subsubsection{Accuracy score}

\subsection{The Ising model}

\section{Implementation}
Implementation is done in python.

\section{Analysis of the methods}

\section{Conclusion}

\bibliography{references}{}
\bibliographystyle{plain}
\end{document}
